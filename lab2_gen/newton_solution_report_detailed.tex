
\documentclass{article}
\usepackage{amsmath}
\begin{document}

\section*{Решение системы нелинейных уравнений методом Ньютона}

Дана система уравнений:
\[
\begin{cases}
\tan(xy) = x^2 \\
0.8x^2 + 2y^2 = 1
\end{cases}
\]

\subsection*{Метод Ньютона}

Метод Ньютона заключается в итерационном уточнении решения с использованием якобиана системы. Якобиан системы:
\[
J(x, y) = \begin{pmatrix}
\frac{\partial f_1}{\partial x} & \frac{\partial f_1}{\partial y} \\
\frac{\partial f_2}{\partial x} & \frac{\partial f_2}{\partial y}
\end{pmatrix}
\]
где
\[
\frac{\partial f_1}{\partial x} = y \cdot \sec^2(xy) - 2x, \quad
\frac{\partial f_1}{\partial y} = x \cdot \sec^2(xy)
\]
\[
\frac{\partial f_2}{\partial x} = 1.6x, \quad
\frac{\partial f_2}{\partial y} = 4y
\]

\subsection*{Итерации}

\subsubsection*{Итерация 1}

Текущее приближение:
\[
x_n = 0.500000, \quad y_n = 0.500000
\]

Значения функций:
\[
f_1(x_n, y_n) = \tan(0.500000 \cdot 0.500000) - (0.500000)^2 = 0.005342
\]
\[
f_2(x_n, y_n) = 0.8 \cdot (0.500000)^2 + 2 \cdot (0.500000)^2 - 1 = -0.300000
\]

Якобиан на текущей итерации:
\[
J(x_n, y_n) = \begin{pmatrix}
-0.467400 & 0.532600 \\
0.800000 & 2.000000
\end{pmatrix}
\]

Решаем систему линейных уравнений для нахождения приращений:
\[
\begin{pmatrix}
-0.467400 & 0.532600 \\
0.800000 & 2.000000
\end{pmatrix}
\begin{pmatrix}
\Delta x_n \\
\Delta y_n
\end{pmatrix}
= -\begin{pmatrix}
0.005342 \\
-0.300000
\end{pmatrix}
\]

Приращения:
\[
\Delta x_n = 0.125260, \quad \Delta y_n = 0.099896
\]

Новое приближение:
\[
x_{n+1} = x_n + \Delta x_n = 0.500000 + 0.125260 = 0.625260
\]
\[
y_{n+1} = y_n + \Delta y_n = 0.500000 + 0.099896 = 0.599896
\]

\subsubsection*{Итерация 2}

Текущее приближение:
\[
x_n = 0.625260, \quad y_n = 0.599896
\]

Значения функций:
\[
f_1(x_n, y_n) = \tan(0.625260 \cdot 0.599896) - (0.625260)^2 = 0.002782
\]
\[
f_2(x_n, y_n) = 0.8 \cdot (0.625260)^2 + 2 \cdot (0.599896)^2 - 1 = 0.032510
\]

Якобиан на текущей итерации:
\[
J(x_n, y_n) = \begin{pmatrix}
-0.557625 & 0.722191 \\
1.000416 & 2.399584
\end{pmatrix}
\]

Решаем систему линейных уравнений для нахождения приращений:
\[
\begin{pmatrix}
-0.557625 & 0.722191 \\
1.000416 & 2.399584
\end{pmatrix}
\begin{pmatrix}
\Delta x_n \\
\Delta y_n
\end{pmatrix}
= -\begin{pmatrix}
0.002782 \\
0.032510
\end{pmatrix}
\]

Приращения:
\[
\Delta x_n = -0.008155, \quad \Delta y_n = -0.010148
\]

Новое приближение:
\[
x_{n+1} = x_n + \Delta x_n = 0.625260 + -0.008155 = 0.617105
\]
\[
y_{n+1} = y_n + \Delta y_n = 0.599896 + -0.010148 = 0.589748
\]

\subsubsection*{Итерация 3}

Текущее приближение:
\[
x_n = 0.617105, \quad y_n = 0.589748
\]

Значения функций:
\[
f_1(x_n, y_n) = \tan(0.617105 \cdot 0.589748) - (0.617105)^2 = 0.000085
\]
\[
f_2(x_n, y_n) = 0.8 \cdot (0.617105)^2 + 2 \cdot (0.589748)^2 - 1 = 0.000259
\]

Якобиан на текущей итерации:
\[
J(x_n, y_n) = \begin{pmatrix}
-0.558897 & 0.706639 \\
0.987368 & 2.358990
\end{pmatrix}
\]

Решаем систему линейных уравнений для нахождения приращений:
\[
\begin{pmatrix}
-0.558897 & 0.706639 \\
0.987368 & 2.358990
\end{pmatrix}
\begin{pmatrix}
\Delta x_n \\
\Delta y_n
\end{pmatrix}
= -\begin{pmatrix}
0.000085 \\
0.000259
\end{pmatrix}
\]

Приращения:
\[
\Delta x_n = 0.000008, \quad \Delta y_n = -0.000113
\]

Новое приближение:
\[
x_{n+1} = x_n + \Delta x_n = 0.617105 + 0.000008 = 0.617113
\]
\[
y_{n+1} = y_n + \Delta y_n = 0.589748 + -0.000113 = 0.589634
\]

\subsection*{Результат}

После нескольких итераций метод Ньютона сходится к решению:
\[
x \approx 0.617105, \quad y \approx 0.589748
\]

\end{document}
