
\documentclass{article}
\usepackage{amsmath}
\begin{document}

\section*{Решение системы нелинейных уравнений методом Ньютона}

Дана система уравнений:
\[
\begin{cases}
\tan(xy) = x^2 \\
0.8x^2 + 2y^2 = 1
\end{cases}
\]

\subsection*{Метод Ньютона}

Метод Ньютона заключается в итерационном уточнении решения с использованием якобиана системы. Якобиан системы:
\[
J(x, y) = \begin{pmatrix}
\frac{\partial f_1}{\partial x} & \frac{\partial f_1}{\partial y} \\
\frac{\partial f_2}{\partial x} & \frac{\partial f_2}{\partial y}
\end{pmatrix}
\]
где
\[
\frac{\partial f_1}{\partial x} = y \cdot \sec^2(xy) - 2x, \quad
\frac{\partial f_1}{\partial y} = x \cdot \sec^2(xy)
\]
\[
\frac{\partial f_2}{\partial x} = 1.6x, \quad
\frac{\partial f_2}{\partial y} = 4y
\]

\subsection*{Итерации}

\begin{tabular}{|c|c|c|c|}
\hline
Итерация & \(x_n\) & \(y_n\) & Норма невязки \\
\hline
0 & 0.500000 & 0.500000 & 0.300048 \\
1 & 0.625260 & 0.599896 & 0.032629 \\
2 & 0.617105 & 0.589748 & 0.000273 \\
3 & 0.617113 & 0.589634 & 0.000000 \\

\hline
\end{tabular}

\subsection*{Результат}

После нескольких итераций метод Ньютона сходится к решению:
\[
x \approx 0.617113, \quad y \approx 0.589634
\]

\end{document}
